\documentclass[10pt]{../usamts}

\realname{Juni Kim}
\username{junikimm}
\usamtsid{38002}
\usamtsyear{35}
\usamtsround{3}

\begin{document}

%%%%%%%%%%%%%%%%%%%%%%%%%%%%%%%%%%%%%
%%%%%%%%%%                 %%%%%%%%%%
%%%%%%%%%%    Problem 1    %%%%%%%%%%
%%%%%%%%%%                 %%%%%%%%%%
%%%%%%%%%%%%%%%%%%%%%%%%%%%%%%%%%%%%%
\begin{solution}

\begin{center}
\begin{asy}
size(8cm);
bool DRAW_SOLUTION = true;

int n = 6;
real LINE_WIDTH = 0.3;
void drawHLine(int x, int y) {
fill((x,y+0.5-LINE_WIDTH/2)--(x,y+0.5+LINE_WIDTH/2)--(x+1,y+0.5+LINE_WIDTH/2)--(x+1,y+0.5-LINE_WIDTH/2)--cycle, gray(0.8));
}
void drawVLine(int x, int y) {
fill((x+0.5-LINE_WIDTH/2,y)--(x+0.5+LINE_WIDTH/2,y)--(x+0.5+LINE_WIDTH/2,y+1)--(x+0.5-LINE_WIDTH/2,y+1)--cycle, gray(0.8));
}
void drawNum(int x, int y, int num) {
label(scale(2.5)*string(num), (x+0.5,y+0.5));
}
void drawSolNum(int x, int y, int num) {
if (DRAW_SOLUTION) {
drawNum(x, y, num);
}
}
drawHLine(2,0);
drawHLine(4,1);
drawHLine(1,2);
drawHLine(3,2);
drawHLine(4,3);
drawHLine(2,4);
drawHLine(3,5);
drawVLine(0,4);
drawVLine(1,3);
drawVLine(2,1);
drawVLine(2,3);
drawVLine(3,4);
drawVLine(4,1);
drawVLine(5,2);

drawNum(0, 0, 5);
drawNum(4, 0, 3);
drawNum(1, 2, 2);
drawNum(3, 3, 4);
int[][] numbers =
{{6,1,3,5,4,2},
{2,6,5,3,1,4},
{3,5,1,4,2,6},
{4,2,6,1,5,3},
{1,3,4,2,6,5},
{5,4,2,6,3,1}};

for(int i = 0; i <= 6; i += 1) {
draw((i,0)--(i,6));
draw((0,i)--(6,i));
}

for(int i=0;i<6; i += 1){
for(int j=0;j<6;j += 1){
if(!(i == 3 & j == 3 | i == 0 & j == 0 | i == 4 & j == 0 | i == 1 & j == 2)){
drawSolNum(i,j,numbers[5-j][i]);
}
}
}
\end{asy}
\end{center}
\end{solution}

%%%%%%%%%%%%%%%%%%%%%%%%%%%%%%%%%%%%%
%%%%%%%%%%                 %%%%%%%%%%
%%%%%%%%%%    Problem 2    %%%%%%%%%%
%%%%%%%%%%                 %%%%%%%%%%
%%%%%%%%%%%%%%%%%%%%%%%%%%%%%%%%%%%%%

\begin{solution}

Without loss of generality, let $a=1$. Then there are no interior cubes, so the problem condition cannot hold. Thus, we can assume $a,b,c \geq 2$.

The number of unpainted cubes is $(a-2)(b-2)(c-2)$. There are clearly $2 \cdot \paren{(a-2)(b-2)+(b-2)(c-2)+(a-2)(c-2)}$ cubes that are painted exactly once, $4 \cdot \paren{(a-2)+(b-2)+(c-2)}$ cubes that are painted exactly twice, and $8$ corners which are each painted thrice.

The problem asks us to find all $a,b,c$ such that $5(a-2)(b-2)(c-2) = abc$, since the probability that an unpainted cube is picked should be 20\%. Without loss of generality, let $a \le b \le c$. Rearranging some of the terms, we get that $\frac{5(a-2)}{a} = \frac{b}{b-2} \cdot \frac{c}{c-2} \le \frac{b^2}{(b-2)^2}$.

\begin{claim}
    For $x \ge 3$, $f(x) = \frac{x-2}{x}$ is strictly increasing.
\end{claim}
\begin{proof}
\[ x < y \implies \frac{2}{x} > \frac{2}{y} \implies 1-\frac{2}{x} < 1 - \frac{2}{y}\]
\end{proof}

\begin{claim}
    For $x \ge 3$, $f(x) = \frac{x^2}{(x-2)^2}$ is strictly decreasing.
\end{claim}
\begin{proof}
Since $\frac{x-2}{x}$ is strictly increasing and strictly less than 1, it follows that $\frac{x}{x-2} > 1$ and is strictly decreasing. Furthermore, $\frac{x^2}{(x-2)^2}$ is also strictly decreasing by this condition.
\end{proof}

We claim that $a$ is upper bounded by 4. Consider if this were not the case; $3 \le \frac{5(a-2)}{a}$ and $\frac{b^2}{(b-2)^2} \le \frac{25}{9} < 3$. So, the above inequality cannot hold, which is a contradiction.

If $a=3$, then we claim that $b \leq 8$. Consider if this were not the case; $\frac{5(a-2)}{a} = \frac{5}{3}$ but $\frac{b^2}{(b-2)^2} \leq \frac{9^2}{7^2} < \frac{5}{3}$, which contradicts our original inequality.

If $a=4$, we claim that $b \leq 5$. Consider if this were not the case; $\frac{5(a-2)}{a} = \frac{5}{2}$ but $\frac{b^2}{(b-2)^2} \leq \frac{6^2}{4^2} < \frac{5}{2}$, which contradicts our original inequality.

Now, we are left with a finite case check:

\begin{center}
\begin{tabular}{c | c | c}
    $(a,b)$ & Required value of $1-\frac{2}{c}$ & Value of $c$\\\hline
    $(3,3)$ & 9/5 & $-\frac{5}{2}$ \\
    $(3,4)$ & 6/5 & $-10$ \\
    $(3,5)$ & 1 & undefined \\
    $(3,6)$ & 9/10 & $20$ (valid) \\
    $(3,7)$ & 21/25 & $\frac{25}{2}$ \\
    $(3,8)$ & 4/5 & $10$ (valid) \\
    $(4,4)$ & 4/5 & $10$ (valid) \\
    $(4,5)$ & 2/3 & $6$ (valid) \\
\end{tabular}
\end{center}

Upon inspection, it is clear that the only possible solutions are the integer triples $(3,6,20)$, $(3,8,10)$, $(4,4,10)$, $(4,5,6)$ and their permutations, which all work.

To answer the question, the number of cubes in the bag must be one of $\fbox{120,160,240,360}$.

\end{solution}


%%%%%%%%%%%%%%%%%%%%%%%%%%%%%%%%%%%%%
%%%%%%%%%%                 %%%%%%%%%%
%%%%%%%%%%    Problem 3    %%%%%%%%%%
%%%%%%%%%%                 %%%%%%%%%%
%%%%%%%%%%%%%%%%%%%%%%%%%%%%%%%%%%%%%
\begin{solution}
We claim that the answer is all $n$ such that \fbox{$3 \nmid n$}.

Let $(a,b)$ denote a game state where we have $a$ ones and $b$ twos. The number of zeroes is irrelevant since they are already out of consideration from our game.
Call a game state \textit{maintainable} if either:
\begin{enumerate}
    \item $a = 1$ and $b \equiv 1 \pmod 3$
    \item $a = 0$ and $b \equiv 0 \pmod 3$
\end{enumerate}
Note that the losing state of $(0,0)$ is \textit{maintainable}. Now, the pith of our solution is the following lemma:
\begin{claim}
    Without loss of generality, let it be Lizzy's turn. If the game state is \textit{maintainable}, Alex, no matter what legal move Lizzy plays, can always make the game state \textit{maintainable} after his turn. Furthermore, Lizzy cannot keep the game state \textit{maintainable} on her turn.
\end{claim}
\begin{proof}
    First, we case bash on Lizzy's moves when the game state is denoted as $(1, 3m+1)$ for some non-negative integer $m$.
    If she decides to delete a two (that is, turn it into a zero), the game state becomes $(1, 3m)$. In this case, Alex can clearly just delete a one to turn the game state into $(0,3m)$.

    Now, there are three different cases where Lizzy chooses to decrement some number of twos, but decides to keep the one that was already there.
    \begin{itemize}
        \item She decrements $3n$ twos; the game state is then $(3n+1, 3(m-n)+1)$. Alex can then decrement $3n$ of the ones (this is possible since $n$ in this particular case is forced to be nonzero) and turn the game state into $(1, 3(m-n)+1)$.
        \item She decrements $3n+1$ twos; the game state is then $\paren{3n+2, 3(m-n)}$. Alex can then decrement all the ones on the board to make the game state $(0,3(m-n))$.
        \item She decrements $3n+2$ twos; the game state is then $\paren{3n+3, 3(m-n)-1}$. Alex can then decrement all of the ones on the board and decrement a two, altering the game step as follows: $(3n+3, 3(m-n)-1) \Rightarrow (0,3(m-n)-1) \Rightarrow (1, 3(m-n)-2)$.
    \end{itemize}
    There are also three different cases (basically the same reasoning) where she deletes the one that was already on the board and decrements some number of twos.
    \begin{itemize}
        \item She decrements $3n$ twos; the game state is then $(3n, 3(m-n)+1)$. Alex can then delete $3n-1$ of the ones and turn the game state into $(1, 3(m-n)+1)$. In the case that $n=0$, Alex can delete a two to turn the game state into $(0, 3(m-n))$.
        \item She decrements $3n+1$ twos; the game state is then $\paren{3n+1, 3(m-n)}$. Alex can then decrement all the ones on the board to make the game state $(0,3(m-n))$.
        \item She decrements $3n+2$ twos; the game state is then $\paren{3n+2, 3(m-n)-1}$. Alex can then decrement all of the ones on the board and decrement one of the twos, altering the game step as follows: $(3n+3, 3(m-n)-1) \Rightarrow (0,3(m-n)-1) \Rightarrow (1, 3(m-n)-2)$.
    \end{itemize}

    Next, we case bash on Lizzy's moves when the game state is denoted as $(0, 3m)$ for some positive integer $m$.
    If she decides to delete a two, the game state becomes $(0, 3m-1)$. Alex can just decrement a two to turn the game state into $(1, 3m-2)$.

    Now, there are three different cases where Lizzy decrements a certain number of twos from the board (this number must be positive):
    \begin{itemize}
        \item She decrements $3n$ twos; the game state is then $(3n, 3(m-n))$. Alex can decrement all the ones and turn the game state into $(0, 3(m-n))$.
        \item She decrements $3n+1$ twos; the game state is then $(3n+1, 3(m-n)-1)$. Alex can decrement all of the ones and decrement a two, turning the game state into $(1, 3(m-n)-2)$.
        \item She decrements $3n+2$ twos; the game state is then $(3n+2, 3(m-n)-2)$. Alex can decrement all but a single one, turning the game state into $(1, 3(m-n)-2)$.
    \end{itemize}

    So, Alex always has a move to ensure the game state is \textit{maintainable}. The case bash also shows that no matter what legal move Lizzy plays, Alex will never have to play on a \textit{maintainable state}.
\end{proof}

Note that the sum of all numbers written on the board is a strictly decreasing monovariant; therefore, all numbers on the board eventually go to zero after a finite number of turns. When this happens, it will be the turn of the player on the \textit{maintainable} state because it can't be otherwise, and so they will lose. As shown before, if the state is \textit{maintainable} on a player's turn, their opponent has a guaranteed strategy to make sure the player's game state is always \textit{maintainable} on their turn.

Now, if $n$ is a multiple of 3, Lizzy starts the game on a \textit{maintainable} game state, so Alex has a winning strategy. Otherwise, Lizzy can make one of the following two moves to force Alex to play on a \textit{maintainable} state:

\begin{itemize}
    \item If $n = 3k+1$; Lizzy can delete a two, turning the game state from $(0, 3k+1)$ to $(0, 3k)$.
    \item If $n = 3k+2$; Lizzy can decrement a two, turning the game state from $(0,3k+2)$ to $(1,3k+1)$.
\end{itemize}

\end{solution}

%%%%%%%%%%%%%%%%%%%%%%%%%%%%%%%%%%%%%
%%%%%%%%%%                 %%%%%%%%%%
%%%%%%%%%%    Problem 4    %%%%%%%%%%
%%%%%%%%%%                 %%%%%%%%%%
%%%%%%%%%%%%%%%%%%%%%%%%%%%%%%%%%%%%%
\begin{solution}

We first solve \textbf{Part a}, for which the statement is \fbox{true}. It is well-known that the cardinality of $\RR^n$ is the same as the cardinality of any open interval $(a,b)$ on the real line. Therefore, there exists some injection which maps $\RR^2$ to $(0,2\pi)$, and then $(0,2\pi)$ to a unique point on the unit circle of the cartesian plane by $x \rightarrow \paren{\sin{x}, \cos{x}}$.
 Any polygon with vertices on this unit circle, if it is not self-intersecting, is immediately convex. Therefore, any polygon on our original plane will get mapped to a convex polygon which is not degenerate (guaranteed because of the injectivity of our map). 
 \qed

We now solve \textbf{Part b}, for which the statement is \fbox{false}.
\begin{claim}
    Let $G^*$ be the image of a convex polygon $G$ under the map $f$ whose points are all on the unit circle. Then, $G^*$ cannot have any convex sub-polygon with more than 3 points. In particular, its convex hull, if it has more than three points, must be a triangle.
\end{claim}
\begin{proof}
    If there exists a convex sub-polygon on $G^*$ with at least 4 points, let it be temporarily denoted $S^*$, we can take the set $C = G \cap (S^*)^{-1}$. Clearly, $\abs{C} \geq 4$, but $C$ and $S^*$ are both convex polygons. So, a convex polygon maps to a convex polygon, contradiction.
\end{proof}

\begin{claim}
    The image of the vertices of a regular 19-gon under the map $f$ must contain at least 6 distinct points.
\end{claim}
\begin{proof}
    Assume for the sake of contradiction that this was not true. Then, there exists a point $x$ such that at least 4 vertices of our 19-gon map to it. Now, consider the set of all vertices $v$ such that $f(v) = x$, which we denote as $V$. The points in $V$ form a convex polygon with at least 4 sides, but the image of $V$ is a single point, which is a degenerate polygon.
\end{proof}

Therefore, we can find points $A$, $B$, $C$, $D$, $E$, and $F$ on the unit circle such that they all map to different points under $f$; let the points they map to be denoted $A^*$, $B^*$, $C^*$, $D^*$, $E^*$, and $F^*$ respectively. Now, the problem is solved with the following lemma:

\begin{claim}
    Six distinct points on the plane, which we will denote $A^*$, $B^*$, $C^*$, $D^*$, $E^*$, and $F^*$, will always form at least one non-concave or degenerate quadrilateral.
\end{claim}
\begin{proof}
\begin{asydef}
void draw_triangledef(pair D, pair E, pair F) {
    label("$D^*$", D, dir(20), p=blue);
    label("$E^*$", E, dir(-50), p=blue);
    label("$F^*$", F, dir(-130), p=blue);
    //draw(D--E--F--cycle);
    draw((2E-D)--(2D-E));
    draw((2F-E)--(2E-F));
    draw((2D-F)--(2F-D));
    
    label("$1$", (5D-E-F)/3, p=fontsize(10pt));
    label("$3$", (5E-F-D)/3, p=fontsize(10pt));
    label("$5$", (5F-D-E)/3, p=fontsize(10pt));
    label("$2$", (3D+3E-2F)/4, p=fontsize(10pt));
    label("$4$", (3E+3F-2D)/4, p=fontsize(10pt));
    label("$6$", (3D+3F-2E)/4, p=fontsize(10pt));
    label("$7$", (D+E+F)/3, p=fontsize(10pt));
}
\end{asydef}
If any three points are collinear, a fourth point will create a triangle (which is a convex simple polygon) or a set of four collinear points (which is not a simple polygon).

\begin{center}
\begin{asy}
size(6cm,0);
defaultpen(fontsize(8pt));

pair D = (0,0), E = (2,0), F = (5,0);
pair A = (4,0);
draw(D--F);

pair[] points = {D, E, F, A};
string[] labels = {"$D^*$", "$E^*$", "$F^*$", "$A^*$"};
int[] dirs = {-90,-90,-90, -90};

for (int i = 0; i < points.length; ++i) {
  label(labels[i], points[i], dir(dirs[i]));
  dot(points[i], red);
}
\end{asy}
\begin{asy}
size(6cm,0);
defaultpen(fontsize(8pt));

pair D = (0,0), E = (2,0), F = (5,0);
pair A = (4,3);
draw(D--F);

pair[] points = {D, E, F, A};
string[] labels = {"$D^*$", "$E^*$", "$F^*$", "$A^*$"};
int[] dirs = {-90,-90,-90, 90};
filldraw(D--F--A--cycle,palecyan,black);

for (int i = 0; i < points.length; ++i) {
  label(labels[i], points[i], dir(dirs[i]));
  dot(points[i], red);
}
\end{asy}
\end{center}

Assuming this is not the case, the six points form a convex hull $\mathcal{H}$ with nonzero area. This convex hull must contain exactly three points; otherwise, the points on the convex hull will determine a convex or degenerate polygon with at least four sides. Without loss of generality, let $A^*$, $B^*$, and $C^*$ be the points on the convex hull. Points $D^*$, $E^*$, and $F^*$, which are all inside $\triangle A^*B^*C^*$, form a triangle, and the extensions of their sides partition the plane into seven convex regions as follows:
\begin{center}
\begin{asy}
size(6cm,0);
import graph;
defaultpen(fontsize(8pt));

pair D = (2,4),E = (0,0),F = (5,0);
draw_triangledef(D,E,F);
\end{asy}
\end{center}

Because $A^*$ is part of the convex hull, it cannot be inside region 7, in which case it would fail to be part of the convex hull. The same reasoning applies for $B^*$ and $C^*$.

If $A^*$ (and likewise for $B^*$ and $C^*$) is on regions 2, 4, or 6, we claim that $D^*E^*F^*A^*$ will be a convex quadrilateral under some permutation. Without loss of generality, let $A^*$ lie on region 2, as shown in the diagram below. Then, points $D^*$ and $E^*$ lie on opposite sides of the line $A^*F^*$. $A^*$ and $F^*$ also lie on opposite sides of the line $E^*D^*$. Therefore, the quadrilateral $E^*A^*D^*F^*$ is not concave.


\begin{center}
\begin{asy}
size(6cm,0);
import graph;
defaultpen(fontsize(8pt));

pair D = (2,4),E = (0,0),F = (5,0);

pair A = (-4,1);
dot(A, red);
fill(F--D--A--E--cycle, palecyan);
label("$A^*$", A, 2*dir(120), p=red);
draw_triangledef(D,E,F);

draw(D--E,red);
draw(A--F,red);
draw(F--D--A--E--cycle, deepgreen);
\end{asy}
\end{center}

We can now assume that points $A^*$, $B^*$, and $C^*$ will only occupy regions 1,3, or 5.

Consider the case where $A^*$, $B^*$, and $C^*$ all reside on one of these regions. Without loss of generality, let all three of them be on region $1$. Then, the line $D^*E^*$ separates $\triangle A^*B^*C^*$ from $F^*$, so $\triangle A^*B^*C^*$ is not a convex hull.

This means that at least two of the regions need to be occupied. Without loss of generality, assume that $A^*$ is in region 1 and $B^*$ is in region 3. Since $D^*$ and $B^*$ are on opposite sides of the line $A^*E^*$, and points $A^*$ and $E^*$ are on opposite sides of the line $B^*D^*$, it follows that $A^*D^*E^*B^*$ is a non-concave quadrilateral.

\begin{center}
\begin{asy}
size(6cm,0);
import graph;
import math;
defaultpen(fontsize(8pt));

pair D = (2,4),E = (0,0),F = (5,0);

pair A = (1,9);
pair B = (1.5,5);
pair C = (3,8.5);
dot(A, red);
dot(B, red);
dot(C, red);
fill(A--B--C--cycle, palecyan);
label("$A^*$", A, dir(120), p=red);
label("$B^*$", B, dir(180), p=red);
label("$C^*$", C, dir(0), p=red);
draw_triangledef(D,E,F);

drawline(D,E,red+linewidth(2));
draw(A--B--C--cycle, deepgreen);
\end{asy}
\begin{asy}
size(6cm,0);
import graph;
import math;
defaultpen(fontsize(8pt));

pair D = (2,4),E = (0,0),F = (5,0);

pair A = (1,6);
pair B = (-2,-2);
dot(A, red);
dot(B, red);
fill(A--D--E--B--cycle, palecyan);
label("$A^*$", A, 2*dir(120), p=red);
label("$B^*$", B, 2*dir(180), p=red);
draw_triangledef(D,E,F);

drawline(A, E, red);
drawline(B, D, red);
draw(A--D--E--B--cycle, deepgreen);

\end{asy}
\end{center}
\end{proof}
\end{solution}

%%%%%%%%%%%%%%%%%%%%%%%%%%%%%%%%%%%%%
%%%%%%%%%%                 %%%%%%%%%%
%%%%%%%%%%    Problem 5    %%%%%%%%%%
%%%%%%%%%%                 %%%%%%%%%%
%%%%%%%%%%%%%%%%%%%%%%%%%%%%%%%%%%%%%
\begin{solution}
The answer is the set of all points that are on the ellipsoid with equation $x^2 + y^2 + 2z^2 = 1$ such that $z \neq 0$.

\begin{center}
\begin{asy}
import three;
unitsize(1cm);
size(200);
//size(6cm,0);
defaultpen(fontsize(6pt));
currentprojection=orthographic(0.75, -3, 1);

triple O = (0,0,0);

triple A = (cos(radians(120)), sin(radians(120)), 0);
triple B = (cos(radians(210)), sin(radians(210)), 0);
triple C = (cos(radians(330)), sin(radians(330)), 0);
triple H = (A + B + C);

triple P = (H.x, H.y, sqrt((1 - abs(H)^2)/2));
triple Pp = (H.x, H.y, -sqrt((1 - abs(H)^2)/2));
label("$H$", align=2N, H);
label("$P$", align=2N, P);
label("$P'$", align=-2N, Pp);
dot(P); dot(Pp); dot(H);
draw(P--Pp, gray(0.2)+dashed);

draw(A--P, gray(0.2)); draw(B--P, gray(0.2)); draw(C--P, gray(0.2));
draw(A--Pp, gray(0.2)); draw(B--Pp, gray(0.2)); draw(C--Pp, gray(0.2));

surface midpoint_sphere(triple x, triple y) {
  return shift((x+y)/2) * scale3(abs((x-y)/2)) * unitsphere;
}

draw(A--B--C--cycle);

real BOUNDARY = 1.8;
draw(O--(BOUNDARY,0,0), gray,Arrow3);
draw(O--(-BOUNDARY,0,0), gray,Arrow3);

draw(O--(0,-BOUNDARY,0), gray,Arrow3);
draw(O--(0,BOUNDARY,0), gray,Arrow3);

draw(O--(0,0,BOUNDARY), gray,Arrow3);
draw(O--(0,0,-BOUNDARY), gray,Arrow3);

surface solution = zscale3(1/sqrt(2)) * unitsphere;
draw(solution, opacity(0.4)+gray,currentlight);
path3 g=(1,0,0)..(0,1,0)..(-1,0,0)..(0,-1,0)..cycle;
draw(g, white);


label("$(0,0,\frac{1}{\sqrt{2}})$", align=2N, (0,0,1/sqrt(2)));
label("$(0,0,-\frac{1}{\sqrt{2}})$", align=-2N, (0,0,-1/sqrt(2)));
dot((0,0,1/sqrt(2)), gray(0.2));
dot((0,0,-1/sqrt(2)), gray(0.2));
label("$A$", align=N, A);
label("$B$", align=N, B);
label("$C$", align=N, C);
\end{asy}
\begin{asy}
import three;
unitsize(1cm);
size(200);
//size(6cm,0);
defaultpen(fontsize(6pt));
currentprojection=orthographic(-0.7, -3, 1);

triple O = (0,0,0);

triple A = (cos(radians(120)), sin(radians(120)), 0);
triple B = (cos(radians(210)), sin(radians(210)), 0);
triple C = (cos(radians(330)), sin(radians(330)), 0);
triple H = (A + B + C);

triple P = (H.x, H.y, sqrt((1 - abs(H)^2)/2));
triple Pp = (H.x, H.y, -sqrt((1 - abs(H)^2)/2));
draw(P--Pp, gray(0.2)+dashed);
label("$H$", align=2N, H);
label("$P$", align=2N, P);
label("$P'$", align=-2N, Pp);
dot(P); dot(Pp); dot(H);

draw(A--P, gray(0.2)); draw(B--P, gray(0.2)); draw(C--P, gray(0.2));
draw(A--Pp, gray(0.2)); draw(B--Pp, gray(0.2)); draw(C--Pp, gray(0.2));

surface midpoint_sphere(triple x, triple y) {
  return shift((x+y)/2) * scale3(abs((x-y)/2)) * unitsphere;
}

draw(A--B--C--cycle);

draw(midpoint_sphere(A,B), opacity(0.25)+red,currentlight);
draw(midpoint_sphere(B,C), opacity(0.25)+green,currentlight);
draw(midpoint_sphere(A,C), opacity(0.25)+blue,currentlight);

real BOUNDARY = 1.8;
draw(O--(BOUNDARY,0,0), gray,Arrow3);
draw(O--(-BOUNDARY,0,0), gray,Arrow3);

draw(O--(0,-BOUNDARY,0), gray,Arrow3);
draw(O--(0,BOUNDARY,0), gray,Arrow3);

draw(O--(0,0,BOUNDARY), gray,Arrow3);
draw(O--(0,0,-BOUNDARY), gray,Arrow3);
path3 g=(1,0,0)..(0,1,0)..(-1,0,0)..(0,-1,0)..cycle;
draw(g, white);

label("$A$", align=N, A);
label("$B$", align=N, B);
label("$C$", align=N, C);
\end{asy}
\end{center}

Let us consider a triangle $\triangle ABC$. For the perpendicularity conditions to hold, $P$ needs to lie on a sphere with radius $\frac{BC}{2}$ and centered at the midpoint of $\overline{BC}$. Let this sphere, excluding the points $B$ and $C$ (since the right angle condition is not well-defined there), be denoted $\Gamma_A$, and define $\Gamma_B$ and $\Gamma_C$ likewise. Note that $\Gamma_A$ is the locus of all points $X$ such that $BX \perp CX$, and likewise for $\Gamma_B$ and $\Gamma_C$. We would like to find the points where all three circles concur.

\begin{center}
\begin{asy}
size(8cm);
import olympiad;
pair A = (3,12), B = (0,0), C = (14,0);
label("$A$", A, dir(80));
label("$B$", B, dir(-90));
label("$C$", C, dir(-70));
draw(A--B--C--cycle);
draw(circle((A+B)/2, abs((A-B)/2)), red);
draw(circle((B+C)/2, abs((B-C)/2)), blue+dashed);
draw(circle((C+A)/2, abs((A-C)/2)), red);

pair D = foot(A,B,C);
pair E = foot(B,A,C);
pair F = foot(C,A,B);
label("$D$", D, dir(-90));
label("$E$", E, dir(20));
label("$F$", F, dir(135));
draw(A--D, red+dashed);
draw(B--E, blue+dashed);
draw(C--F, blue+dashed);
draw(rightanglemark(A,D,B, s=8));
pair H = orthocenter(A,B,C);
label("$H$", H, dir(20));

pair M = (B+C)/2;
label("$M$", M, dir(-90));
dot(M);
\end{asy}
\begin{asy}
import three;
unitsize(1cm);
size(250);
//size(6cm,0);
defaultpen(fontsize(8pt));
currentprojection=TopView;

triple O = (0,0,0);

triple A = (cos(radians(120)), sin(radians(120)), 0);
triple B = (cos(radians(210)), sin(radians(210)), 0);
triple C = (cos(radians(330)), sin(radians(330)), 0);
triple H = (A + B + C);

triple P = (H.x, H.y, sqrt((1 - abs(H)^2)/2));
triple Pp = (H.x, H.y, -sqrt((1 - abs(H)^2)/2));
/*
label("$H$", align=2N, H);
label("$P'$", align=-2N, Pp);
dot(Pp); dot(H);
*/
label("$P$", align=2N, P);
dot(P);

draw(A--P, gray(0.2)); draw(B--P, gray(0.2)); draw(C--P, gray(0.2));
///draw(A--Pp, gray(0.2)); draw(B--Pp, gray(0.2)); draw(C--Pp, gray(0.2));

surface midpoint_sphere(triple x, triple y) {
  return shift((x+y)/2) * scale3(abs((x-y)/2)) * unitsphere;
}

draw(A--B--C--cycle);

draw(midpoint_sphere(A,B), opacity(0.3)+red,currentlight);
draw(midpoint_sphere(B,C), opacity(0.3)+green,currentlight);
draw(midpoint_sphere(A,C), opacity(0.3)+blue,currentlight);

real BOUNDARY = 1.8;
draw(O--(BOUNDARY,0,0), gray,Arrow3);
draw(O--(-BOUNDARY,0,0), gray,Arrow3);

draw(O--(0,-BOUNDARY,0), gray,Arrow3);
draw(O--(0,BOUNDARY,0), gray,Arrow3);

draw(O--(0,0,BOUNDARY), gray,Arrow3);
draw(O--(0,0,-BOUNDARY), gray,Arrow3);

path3 g=(1,0,0)..(0,1,0)..(-1,0,0)..(0,-1,0)..cycle;
draw(g, white);

label("$A$", align=N, A);
label("$B$", align=N, B);
label("$C$", align=N, C);
\end{asy}
\end{center}

Because of this definition, $\Gamma_B$ and $\Gamma_C$ will pass through $A$ and the foot of the perpendicular from $A$ to $BC$ (which we will temporarily denote $D$). Therefore, the locus of all points that are on both $\Gamma_B$ and $\Gamma_C$ will be a circle with diameter $AD$ and which is perpendicular to the xy plane, excluding the point $A$. So, the projection of $P$ onto the xy plane, mapping from $(x,y,z) \rightarrow (x,y,0)$, must be on the set of all points on segment $AD$ excluding $A$. We repeat the same reasoning for the other two cases. Therefore, the projection of $P$ is on the orthocenter of $\triangle ABC$ (denoted $H$), and it can only exist if $\triangle ABC$ is acute.

\begin{claim}
All points on the open unit disk are an orthocenter of some acute triangle with vertices on the unit circle.
\end{claim}
\begin{proof}
\begin{figure}[htbp]
\centering
\begin{asy}
size(5cm);
import graph;
import math;
import olympiad;

pair A,B,C;
A = 3*dir(90);
B = 3*dir(180+50);
C = 3*dir(-50);

label("$A$", A, dir(135));
label("$B$", B, dir(210));
label("$C$", C, dir(-30));

draw(A--B--C--cycle);
pair O = circumcenter(A,B,C);
label("$O$", O, dir(135));
dot(O);
pair H = orthocenter(A,B,C);
label("$H$", H, dir(135), deepgreen);
dot(H, deepgreen);

pair D = intersectionpoint(C--A,B--(10H-9B));
dot(D);
drawline(B,D,p=gray);
draw(rightanglemark(H,D,C));

drawline(H,O, p=gray);
draw(circumcircle(A,B,C),blue);

pair M = (B+C)/2;

label("$M$", M, dir(-45));
dot(M);
draw(rightanglemark(B,M,H, s=8));

pair Hp = reflect(B,C)*H;
dot(Hp);
label("$H'$", Hp, dir(-45));

add(pathticks(H--M,n=1, s=5));
add(pathticks(Hp--M,n=1, s=5));

add(pathticks(B--M,n=2, s=5));
add(pathticks(C--M,n=2, s=5));
\end{asy}
\end{figure}
    Let the distance from $O$ to $H$ be denoted $r$. Extend ray $OH$ (if $O=H$, any ray from $O$ will suffice) to meet the unit circle again at $H'$. Let $M$ be the midpoint of $HH'$. Let $A$ be the antipode of $H'$ and $B$ and $C$ the intersections of the perpendicular bisector of $HH'$ with the unit circle. Since $O$ is inside our triangle, it follows that $\triangle ABC$ is acute. By Power of Point, $H'M \cdot AM = HM \cdot AM = MB \cdot MC$. This implies that $\frac{MB}{HM} = \frac{AM}{MC}$, and so $\triangle AMC \sim \triangle BMH$. Since $HM \perp MC$, it follows that $BH \perp AC$. The same reasoning applies on the other side by symmetry, so $H$ is the orthocenter of $\triangle ABC$.
\end{proof}

\begin{claim}
    In directed lengths (positive z-direction), $HP = \pm \sqrt{\frac{1 - OH^2}{2}}$. So, the $z$-coordinate of $P$ is only dependent on $H$ and not on $\triangle ABC$.
\end{claim}

\begin{proof}
\begin{figure}[htbp]
\centering
\begin{asy}
import olympiad;
size(6cm);
defaultpen(fontsize(8pt));

pair A,B,C,D,E,F,H,O,M,Mp,Hp;
A = (3,12); B = (0,0); C = (14,0);
D = foot(A,B,C); E = foot(B,A,C); F = foot(C,A,B); H = orthocenter(A,B,C);
O = circumcenter(A,B,C);
Mp=(C+F)/2;
M=(B+C)/2;
Hp = 2*F-H;

dot(O); dot(Mp, blue); dot(M);
draw(A--B--C--cycle);
draw(A--D, dashed); draw(B--E, dashed); draw(C--Hp,blue);
draw(circumcircle(A,B,C));

draw(M--H,red);
draw(M--C,red);
draw(M--Mp,red+dashed);
draw(rightanglemark(M,Mp,C));
draw(circumcircle(B,C,F), blue+dashed);
draw(circle((F+C)/2, abs((F-C)/2)), red+dashed);

add(pathticks(H--F, n=1));
add(pathticks(Hp--F, n=1));

real[] angles = {80, -90, -70, -70, 90, 120,45,130,-70,80,135};
pair[] points = {A,B,C,D,E,F,O,H,M,Mp, Hp};
string[] labels = {"A", "B", "C", "D", "E", "F", "O", "H", "M", "M'", "H'"};

for (int i = 0; i < angles.length; ++i) {
  label("$" + labels[i] + "$", points[i], dir(angles[i]));
}
\end{asy}
\end{figure}

Let $M$ be the midpoint of $BC$. Since $M$ is the center of $\Gamma_A$ and $P$ is the concurrency point of all three circles, $MP = MB = MC$. Therefore, $HP^2 = MC^2 - MH^2$. By Power of Point,
$$\frac{1 - OH^2}{2} = \frac{OA^2 - OH^2}{2} = \frac{CH \cdot HH'}{2} = CH \cdot HF$$ Now, Let $M'$ be the midpoint of $CF$. Since $CF$ is a chord of the circumcircle of $BFEC$ with center $M$, $MM' \perp CF$. By the Pythagorean theorem and a little bit of algebra,
\[
MC^2 - MH^2 = \paren{MM'^2 + M'C^2} - \paren{MM'^2 + M'H^2} = M'C^2 - M'H^2 \]
This value is by definition equal to $-\Pow_{\omega}(H)$, where $\omega$ is the circle with diameter $FC$. So, $M'C^2 - M'H^2 = FH \cdot CH$. This reasoning can be applied to the other spheres $\Gamma_B$ and $\Gamma_C$, so the $P$ we have derived is the concurrency point of all three spheres.

\end{proof}

\end{solution}

\end{document}